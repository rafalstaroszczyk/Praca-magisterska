\section{Ciało o dwóch stanach}
Stany ciała są oznaczone \(g\) dla stanu podstawowego oraz \(e\) dla stanu wzbudzonego. Przejścia między stanami są możliwe na trzy sposoby: absorpcja kwantu pola elektromagnetycznego o energii \(E_f = h\nu = E_e - E_g\) przez ciało w stanie podstwowym; emisja wymuszona tym samym kwantem ze stanu wzbudzonego na podstawowy z emisją dodatkowego kwantu; oraz emisja spontaniczna ze stanu wzbudzonego na podstawowy z emisją tego samego kwantu. Dynamika populacji obu stanów wyrażona jest wzorami:
\begin{align}
	\odv{n_g}{t} &=   \pab{A_{eg} + B_{eg} u}n_e - B_{ge} u n_g, \\
	\odv{n_e}{t} &= - \pab{A_{eg} + B_{eg} u}n_e + B_{ge} u n_g,
\end{align}
gdzie \(A_{eg}\), \(B_{eg}\), \(B_{ge}\) to współczynniki Einsteina określające odpowiednio emisję spontaniczną, emisję wymuszoną oraz wzbudzenie. Szybkość drugiego i trzeciego z tych procesów jest zależna od gęstości promieniowania elektromagnetycznego, ponieważ zachodzą one pod wpływem interakcji z istniejącymi już kwantami. W stanie stacjonarnym mamy \(\odv{n_g}{t} = \odv{n_g}{t} = 0\), co prowadzi do równania:
\begin{gather}
	  \pab{A_{eg} + B_{eg} u}n_e - B_{ge} u n_g = 0 \\
	- \pab{A_{eg} + B_{eg} u}n_e + B_{ge} u n_g = 0 \\
	A_{eg} n_e + B_{eg} u n_e = B_{ge} u n_g \\
	\pab{A_{eg} + B_{eg} u} n_e = B_{ge} u n_g \\
	\frac{n_e}{n_g} = \frac{B_{ge} u}{A_{eg} + B_{eg} u} \\
	\frac{n_e}{n_g} = \frac{\frac{B_{ge}}{B_{eg}}}{\frac{A_{eg}}{B_{eg}}\inv{u} + 1} \\
	u = \frac{2h\nu^3}{c^2}\inv{\exp\pab{\frac{h\nu}{kT}} - 1} \\
	\inv{u} = \frac{c^2}{2h\nu^3}\bab{\exp\pab{\frac{h\nu}{kT}} - 1} \\
	\frac{n_e}{n_g} = \frac{\frac{B_{ge}}{B_{eg}}}{\frac{A_{eg}}{B_{eg}}\inv{u} + 1} \\
\end{gather}

Współczynniki Einsteina:
\begin{gather*}
	\odv{n_1}{t} = - B_{12} u n_1 + \pab{A_{21} + B_{21} u} n_2 \\
	\odv{n_2}{t} =   B_{12} u n_1 - \pab{A_{21} + B_{21} u} n_2 \\
	\frac{n_1}{n} = \frac{g_1}{Z}\exp\pab{-\frac{E_1}{kT}} \\
	\frac{n_2}{n} = \frac{g_2}{Z}\exp\pab{-\frac{E_2}{kT}} \\
	\frac{n}{n_2} = \frac{Z}{g_2}\exp\pab{\frac{E_2}{kT}} \\
	\frac{n_1}{n_2} = \frac{g_1}{g_2}\exp\pab{\frac{E_2 - E_1}{kT}} = \frac{g_1}{g_2}\exp\pab{\frac{h\nu}{kT}} \\
	A_{21} + B_{21} u = B_{12} u \frac{n_1}{n_2} \\
	A_{21}g_{2} \exp\pab{-\frac{h\nu}{kT}} + B_{21}g_{2} \exp\pab{-\frac{h\nu}{kT}} u = B_{12}g_{1} u \\
	u = \frac{\frac{2h\nu^3}{c^3}}{\exp\pab{\frac{h\nu}{kT}} - 1} \\
	A_{21}g_{2} \exp\pab{-\frac{h\nu}{kT}} + B_{21}g_{2} \exp\pab{-\frac{h\nu}{kT}} \frac{\frac{2h\nu^3}{c^3}}{\exp\pab{\frac{h\nu}{kT}} - 1} = B_{12}g_{1} \frac{\frac{2h\nu^3}{c^3}}{\exp\pab{\frac{h\nu}{kT}} - 1} \\
	A_{21}g_{2} \exp\pab{-\frac{h\nu}{kT}} \pab{\exp\pab{\frac{h\nu}{kT}} - 1} + B_{21}g_{2} \exp\pab{-\frac{h\nu}{kT}} \frac{2h\nu^3}{c^3} = B_{12}g_{1} \frac{2h\nu^3}{c^3} \\
	A_{21}g_{2} \bab{1 - \exp\pab{-\frac{h\nu}{kT}}} + B_{21}g_{2} \exp\pab{-\frac{h\nu}{kT}} \frac{2h\nu^3}{c^3} = B_{12}g_{1} \frac{2h\nu^3}{c^3} \\
	T \rightarrow 0 \quad\Rightarrow\quad 
	A_{21}g_{2} = B_{12}g_{1} \frac{2h\nu^3}{c^3} \\
	T \rightarrow \infty \Rightarrow 
	B_{21}g_{2} = B_{12}g_{1} \\
	B_{12} = B_{21}\frac{g_{2}}{g_{1}} \\
	\frac{A_{21}}{B_{21}} = \frac{2h\nu^3}{c^3} \\
	\frac{B_{21}}{B_{12}} = \frac{g_1}{g_2}
\end{gather*}