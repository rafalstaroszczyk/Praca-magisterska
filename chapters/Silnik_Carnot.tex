\section{Silnik Carnota}
Cykl Carnota jest serią czterech przemian, które tworzą cykl zamknięty. Przemiany te to po kolei: sprężanie izotermiczne w temperaturze $T_1$; sprężanie adiabatyczna, której skutkiem jest podniesienie temperatury do $T_2$; rozprężanie izotermiczne w tej samej temperaturze; oraz rozprężanie adiabatyczne do temperatury początkowej. Zakładając, że w przemianach adiabatycznych nie występują żadne straty w postaci na przykład tarcia, a przemiany izotermiczne występują bez różnicy temperatur między czynnikiem roboczym a zbiornikiem cieplnym. Ponieważ energia wewnętrzna $U$ jest całką zupełną to w procesie kołowym:
\begin{gather}
\d{U} = \idif{Q} - \idif{W},
W = \oint \idif{W} = \oint \idif{Q} = Q.
\end{gather}
Rozdzielając ciepło na dostarczone $Q_2$ i wyprowadzone $Q_1$ mamy
\begin{equation}
W = Q_2 - Q_1.
\end{equation}
Opisany powyżej cykl jest cyklem prawobieżnym, którego skutkiem jest wykonanie pracy $W = \oint p \d{V}$ kosztem ciepła $Q_2$ pobranego ze zbiornika ciepła w temperaturze $T_2$. 
Sprawność silnika cieplnego jest zdefiniowana jako
\begin{equation}\label{eq:def_sprawnosc_silnika}
\eta = \frac{W}{Q_2} = \frac{Q_2 - Q_1}{Q_2} = 1 - \frac{Q_1}{Q_2}.
\end{equation}
Każda z przemian silnika Carnota jest odwracalna, a więc i cały cykl jest odwracalny. Silnik taki można uruchomić w drugą stronę. Spowoduje to zmianę znaków $W$, $Q_2$ oraz $Q_1$ i w efekcie kosztem włożonej pracy pobrane zostanie ciepło $Q_1$ z zimniejszego $T_1$ i oddanie ciepła $Q_2$ to cieplejszego $T_2$. Aby przemiana izotermiczna była odwracalna musi być nieskończenie powolna. Cykl ten jest więc nieosiągalny w rzeczywistości, ale ma duże znaczenie w teorii termodynamiki, ponieważ stanowi górną granicę dla sprawności silników cieplnych.

\begin{figure}[H]
	\centering
	\begin{subfigure}{0.45\textwidth}
		\centering
		\begin{tikzpicture}
	\begin{axis}[xlabel=$V$,
		ylabel=$p$,
		axis x line=bottom,
		axis y line=middle,
		xmin=0,
		xmax=1.2,
		ymin=0,
		ymax=1.2,
		ticks=none,
		every axis x label/.style={at={(current axis.right of origin)},anchor=north east},
		every axis y label/.style={at={(current axis.above origin)},anchor=north east},
		declare function={
			r = 2;
			tau = 2;
			f = 3;
			kappa = 1 + 2/f;
			V1 = 1;
			V2 = 1/r;
			V3 = 1/r * tau^(-1/(kappa-1));
			V4 = tau^(-1/(kappa-1));
			p1 = 1/r * tau^(-kappa/(kappa-1));
			p2 = tau^(-kappa/(kappa-1));
			p3 = 1;
			p4 = 1/r;
		}]
		%\addplot[mark=none] {x^2/4};
		%\addplot[mark=none] {-1};
		\addplot[mark=*] coordinates {(V1,p1)} node[anchor=north west]{$1$};
		\addplot[mark=*] coordinates {(V2,p2)} node[anchor=north east]{$2$};
		\addplot[mark=*] coordinates {(V3,p3)} node[anchor=south east]{$3$};
		\addplot[mark=*] coordinates {(V4,p4)} node[anchor=south west]{$4$};
		%\addplot[domain=V2:V1] {1/r * tau^(-kappa/(kappa-1)) / x};
		%\addplot[domain=V3:V2] {1/r^kappa * tau^(-kappa/(kappa-1)) / x^kappa};
		%\addplot[domain=V3:V4] {1/r * tau^(-1/(kappa-1)) / x};
		%\addplot[domain=V4:V1] {1/r * tau^(-kappa/(kappa-1)) / x^kappa};
		\coordinate (A) at (V1,p1);
		\coordinate (B) at (V2,p2);
		\coordinate (C) at (V3,p3);
		\coordinate (D) at (V4,p4);
		\draw[black, thick] (A) -- (B);
	\end{axis}
\end{tikzpicture}

		\caption{}
		\label{fig:pv_silnik_carnot}
	\end{subfigure}
	\begin{subfigure}{0.45\textwidth}
		\centering
		\begin{tikzpicture}
	\begin{axis}[width=\textwidth,
		xlabel=$V$,
		ylabel=$p$,
		axis x line=bottom,
		axis y line=middle,
		xmin=0,
		xmax=1.2,
		ymin=0,
		ymax=1.2,
		ticks=none,
		every axis x label/.style={at={(current axis.right of origin)},anchor=north east},
		every axis y label/.style={at={(current axis.above origin)},anchor=north east},
		declare function={
			r = 2;
			tau = 2;
			f = 3;
			kappa = 1 + 2/f;
			V1 = 1/r;
			V2 = 1;
			p1 = 1/r^kappa * 1/tau;
			p2 = 1/tau;
			p3 = 1;
			p4 = 1/r^kappa;
		}]
		\addplot[mark=*] coordinates {(V2,p1)} node[anchor=north west]{$1$};
		\addplot[mark=*] coordinates {(V1,p2)} node[anchor=north east]{$2$};
		\addplot[mark=*] coordinates {(V1,p3)} node[anchor=south east]{$3$};
		\addplot[mark=*] coordinates {(V2,p4)} node[anchor=south west]{$4$};
		\addplot[domain=V1:V2] {1/r^kappa * 1/tau / x^kappa} node [pos=0.5, below, sloped] {$Q=0$};
		\addplot[mark=none] coordinates {(V1, p2) (V1, p3)};
		\addplot[domain=V1:V2] {1/r^kappa / x^kappa}         node [pos=0.5, above, sloped] {$Q=0$};
		\addplot[mark=none] coordinates {(V2, p4) (V2, p1)};
	\end{axis}
\end{tikzpicture}

		\caption{}
		\label{fig:pv_silnik_otto}
	\end{subfigure}
	\captionsetup{subrefformat=parens}
	\caption{Porównanie wykresów cykli prawobieżnych: \subref{fig:pv_silnik_carnot} Carnota i \subref{fig:pv_silnik_otto} Otto.}
\end{figure}