\section{Silnik Carnota}
Cykl Carnota jest serią czterech przemian, które tworzą cykl zamknięty. Przemiany te to po kolei: sprężanie izotermiczne w temperaturze $T_1$; sprężanie adiabatyczna, której skutkiem jest podniesienie temperatury do $T_2$; rozprężanie izotermiczne w tej samej temperaturze; oraz rozprężanie adiabatyczne do temperatury początkowej. Zakładając, że w przemianach adiabatycznych nie występują żadne straty w postaci na przykład tarcia, a przemiany izotermiczne występują bez różnicy temperatur między czynnikiem roboczym a zbiornikiem cieplnym. Ponieważ energia wewnętrzna $U$ jest całką zupełną to w procesie kołowym:
\begin{gather}
\d{U} = \idif{Q} - \idif{W},
W = \oint \idif{W} = \oint \idif{Q} = Q.
\end{gather}
Rozdzielając ciepło na dostarczone $Q_2$ i wyprowadzone $Q_1$ mamy
\begin{equation}
W = Q_2 - Q_1.
\end{equation}
Opisany powyżej cykl jest cyklem prawobieżnym, którego skutkiem jest wykonanie pracy $W = \oint p \d{V}$ kosztem ciepła $Q_2$ pobranego ze zbiornika ciepła w temperaturze $T_2$. 
Sprawność silnika cieplnego jest zdefiniowana jako
\begin{equation}\label{eq:def_sprawnosc_silnika}
\eta = \frac{W}{Q_2} = \frac{Q_2 - Q_1}{Q_2} = 1 - \frac{Q_1}{Q_2}.
\end{equation}
Każda z przemian silnika Carnota jest odwracalna, a więc i cały cykl jest odwracalny. Silnik taki można uruchomić w drugą stronę. Spowoduje to zmianę znaków $W$, $Q_2$ oraz $Q_1$ i w efekcie kosztem włożonej pracy pobrane zostanie ciepło $Q_1$ z zimniejszego $T_1$ i oddanie ciepła $Q_2$ to cieplejszego $T_2$. Aby przemiana izotermiczna była odwracalna musi być nieskończenie powolna. Cykl ten jest więc nieosiągalny w rzeczywistości, ale ma duże znaczenie w teorii termodynamiki, ponieważ stanowi górną granicę dla sprawności silników cieplnych.
