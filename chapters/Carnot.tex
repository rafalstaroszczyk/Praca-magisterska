\chapter{Termodynamika klasyczna}

\section{Pierwsza Zasada Termodynamiki}

\section{Druga Zasada Termodynamiki}

\section{Silnik Carnota}
Cykl Carnota jest serią czterech przemian, które tworzą cykl zamknięty. Przemiany te to po kolei: sprężanie izotermiczne w temperaturze $T_1$; sprężanie adiabatyczna, której skutkiem jest podniesienie temperatury do $T_2$; rozprężanie izotermiczne w tej samej temperaturze; oraz rozprężanie adiabatyczne do temperatury początkowej. Zakładając, że w przemianach adiabatycznych nie występują żadne straty w postaci na przykład tarcia, a przemiany izotermiczne występują bez różnicy temperatur między czynnikiem roboczym a zbiornikiem cieplnym. Ponieważ energia wewnętrzna $U$ jest całką zupełną to w procesie kołowym:
\begin{gather}
\d{U} = \idif{Q} - \idif{W},
W = \oint \idif{W} = \oint \idif{Q} = Q.
\end{gather}
Rozdzielając ciepło na dostarczone $Q_2$ i wyprowadzone $Q_1$ mamy
\begin{equation}
W = Q_2 - Q_1.
\end{equation}
Opisany powyżej cykl jest cyklem prawobieżnym, którego skutkiem jest wykonanie pracy $W = \oint p \d{V}$ kosztem ciepła $Q_2$ pobranego ze zbiornika ciepła w temperaturze $T_2$. 
Sprawność silnika cieplnego jest zdefiniowana jako
\begin{equation}\label{eq:def_sprawnosc_silnika}
\eta = \frac{W}{Q_2} = \frac{Q_2 - Q_1}{Q_2} = 1 - \frac{Q_1}{Q_2}.
\end{equation}
Każda z przemian silnika Carnota jest odwracalna, a więc i cały cykl jest odwracalny. Silnik taki można uruchomić w drugą stronę. Spowoduje to zmianę znaków $W$, $Q_2$ oraz $Q_1$ i w efekcie kosztem włożonej pracy pobrane zostanie ciepło $Q_1$ z zimniejszego $T_1$ i oddanie ciepła $Q_2$ to cieplejszego $T_2$. 

\section{Twierdzenie Carnota}
Udowodnimy teraz, że dla danych dwóch zbiorników ciepła o temperaturach zimniejszego $T_1$ i cieplejeszego $T_2$ silnik odwracalny zawsze będzie miał maksymalną sprawność. Przyjmijmy, że mamy jeden silnik odwracalny, może być to silnik Carnota, pracujący w obiegu lewobieżnym, który oznaczymy C, oraz dowolny inny silnik X pracujący w obiegu prawobieżnym. Mają one sprawności odpowiednio $\eta_C$ oraz $\eta_X$. Są one ze sobą połączone tak, że cała praca wytworzona przez silnik X jest wykorzystywana przez silnik C. Niech ciepło dostarczone do zbiornika ciepłego przez silnik C jest równa $Q$. Wtedy praca wytworzona przez X będzie według równania \eqref{eq:def_sprawnosc_silnika} równa $\eta_C Q$, a ciepło pobrane przez silnik C ze zbiornika zimnego będzie równe $\pab{1 - \eta_C}Q$. Dla silnika X ciepło pobrane ze zbiornika ciepłego i oddane do zimnego są równe odpowiednio $\frac{W}{\eta_X} = \frac{\eta_C}{\eta_X}Q$ oraz $\pab{\frac{\eta_C}{\eta_X} - \eta_C}Q$. Całkowite ciepło pobrane ze zbiornika ciepłego i oddane do zbiornika zimnego jest równe $\pab{\frac{\eta_C}{\eta_X} - 1}Q$. Z drugiej zasady termodynamiki wiemy, że niemożliwym jest, aby w jakimkolwiek procesie jedynym skutkiem było przepłynięcie ciepła ze zbiornika zimniejszego do cieplejszego. Stąd, ponieważ $Q > 0$ otrzymujemy $\pab{\frac{\eta_C}{\eta_X} - 1} \geq 1$, co sprowadza się do:
\begin{equation}\label{eq:tw_carnot}
\eta_C \geq \eta_X.
\end{equation}
Sprawność dowolnego silnika nie może być większa od sprawności silnika odwracalnego. W szczególności sprawność każdego silnika odwracalnego jest sobie równa i jest zależna jedynie od temperatur zbiorników;
\begin{equation}
\eta = 1 - \frac{Q_1}{Q_2} = 1 - f\pab{\theta_1, \theta_2}.
\end{equation}

